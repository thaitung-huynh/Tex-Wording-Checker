\chapter{Anforderungsanalyse}
\label{sec:anforderungsanalyse}

Um wissenschaftlich korrekt zu erfassen, was konzipiert und umgesetzt werden soll, werden Anforderungen erstellt. Am einfachsten ist dies anhand verschiedener Szenarien, die den Ist-Zustand beschreiben. Um zum Soll-Zustand zu gelangen, müssen bestimmte Anforderungen, die Sie aufstellen, erfüllt werden. Durch die Verwendung mehrerer Szenarien sind diese Anforderungen nicht auf ein bestimmtes Szenario zugeschnitten, sondern möglichst allgemeingültig. Andere Varianten sind eine allgemeine Herleitung und basierend auf Literatur.

Je nachdem wie viele Anforderungen Sie aufstellen, kann es notwendig sein, diese zu gruppieren (z.B. funktionale und nicht-funktionale Anforderungen) und/oder zu gewichten (z.B. MUSS, SOLL, KANN). Praktisch ist eine Anforderungstabelle am Ende des Kapitels, welche alle Anforderungen gesammelt zeigt. Diese Tabelle lässt sich leicht weiterverwenden.

Im nächsten Kapitel vergleichen Sie die aufgestellten Anforderungen mit der Literatur. Diese kann einzelne Anforderungen vollständig, teilweise oder gar nicht erfüllen. Das herausgearbeitete Delta begründet Ihre Arbeit. Abschließend müssen die Anforderungen mit der eigenen Lösung im Kapitel Evaluation verglichen werden.
